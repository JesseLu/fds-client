\documentclass{article}
\newcommand{\I}[1]{\item[\texttt{ #1 }]}
\newcommand{\T}{\texttt}
\title{lightlabsFDS File Format}
\begin{document}
\maketitle
\tableofcontents

\begin{abstract}
lightlabsFDS uses the HDF5 file format as a low-level interface. 
HDF5 files are used to store both the input and output parameters of a 
    simulation.
This specification is particularly useful in determining whether a feature
    or function is the responsibility of the client or server.
\end{abstract}

\section{General remarks}
\begin{itemize}
\item   All floating-point values are stored in single-precision, and all
            three-dimensional fields are subject to lossless compression.
\item   Be aware that Matlab arrays are stored in column-major ordering;
            however, HDF5 uses row-major ordering. 
        To convert a three-dimensional array from column-major to row-major
            ordering in Matlab, one can use the \T{permute} function.
\end{itemize}
The following work-distribution is implied from the input and output
    file formats.
Specifically, the server is only responsible for computing the output 
    $E$-fields based on the information found in the input file.
Consequently, the client is then responsible for features such as
\begin{itemize}
\item   setting up PML parameters,
\item   taking care of symmetry planes, 
\item   shape generation (including edge-smoothing),
\item   any mode-solving necessary for determing input sources,
\item   visualizing input and output fields,
\item   calculating $D$-, $H$-, and $B$- output fields, and 
\item   interpreting those output fields (calculating flux, energy, \ldots).
\end{itemize}

\section{Input file format}
The input HDF5 file consists of the following fields:
\begin{description}
\I  {reservation\_id} 
    An identifier of some kind.
    Some kind of random number that certifies that this is indeed from the
        correct user.
    The purpose is to reserve the user's place in line and to authenticate
        the user as well.

\I  {omega}
    Floating-point scalar which determines the angular frequency
        of the simulation.

\I  {shape} 
    A 3-element array of positive integers in the format 
        \T{xx} $\times$ \T{yy} $\times$ \T{zz}.

\I  {boundary\_conditions}
    6-element array of character strings which denote the boundary conditions 
        at the endfaces of the grid in the order 
        $-x$, $+x$, $-y$, $+y$, $-z$, and $+z$.
    Each element must be one of three character strings: 
        \T{pec} (perfect electrical conductor), 
        \T{pmc} (perfect magnetic conductor), or 
        \T{per} (periodic).

\I  {sx\_real, sx\_imag, sy\_real, sy\_imag, sz\_real, sz\_imag}
    One-dimensional floating-point arrays describing the stretched-coordinate
        distances between adjacent grid points.
    Specifically, these denote the distances \emph{between} integer points 
        along the $x$-, $y$-, and $z$-directions 
        (e.g. distance between $x = 0$ and $x = 1$).
    The length of the \T{si\_real} or \T{si\_imag} array,
        where \T{i = x,y,z},
        are \T{xx}, \T{yy}, and \T{zz} respectively.
    Lastly, the first element of the array corresponds to the distance between
        the points at positions \T{0} and \T{1},
        while the last element of the array corresponds to the distance between
        the points at positions \T{N} and (hypothetical) \T{N+1}.

\I  {tx\_real, tx\_imag, ty\_real, ty\_imag, tz\_real, tz\_imag}
    One-dimensional floating-point arrays describing the stretched-coordinate
        distances between adjacent grid points.
    Specifically, these denote the distances \emph{centered} at integer points 
        along the $x$-, $y$-, and $z$-directions
        (e.g. distance between $x = 0$ and $x = 1$)
    The length of the \T{ti\_real} or \T{ti\_imag} array,
        where \T{i = x,y,z},
        are \T{xx}, \T{yy}, and \T{zz} respectively.
    Lastly, the first element of the array corresponds to the distance between
        the points at positions \T{-0.5} and \T{0.5},
        while the last element of the array corresponds to the distance between
        the points at positions \T{N-0.5} and (hypothetical) \T{N+0.5}.

\I  {ex\_real, ex\_imag, ey\_real, ey\_imag, ez\_real, ez\_imag}
    Three-dimensional floating-point arrays describing the values of $\epsilon$.
    The shape of the arrays must be \T{xx} $\times$ \T{yy} $\times$ \T{zz}.

\I  {mx\_real, mx\_imag, my\_real, my\_imag, mz\_real, mz\_imag}
    Three-dimensional floating-point arrays describing the values of $\mu$.
    The shape of the arrays must be \T{xx} $\times$ \T{yy} $\times$ \T{zz}.

\I  {Jx\_real, Jx\_imag, Jy\_real, Jy\_imag, Jz\_real, Jz\_imag}
    Three-dimensional floating-point arrays describing the values of the
        current sources.
    The shape of the arrays must be \T{xx} $\times$ \T{yy} $\times$ \T{zz}.
\end{description}

\section{Output file format}
The output HDF5 file that is returned to the user
    consists of the following fields:
\begin{description}
\I  {Ex\_real, Ex\_imag, Ey\_real, Ey\_imag, Ez\_real, Ez\_imag}
    Three-dimensional floating-point arrays describing the values of the
        current sources.
    The shape of the arrays is \T{xx} $\times$ \T{yy} $\times$ \T{zz}.
\end{description}

\end{document}
